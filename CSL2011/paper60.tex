\documentclass[a4paper,UKenglish]{lipics}

\usepackage{microtype}

\bibliographystyle{plain}% the recommended bibstyle

% Author macros %%%%%%%%%%%%%%%%%%%%%%%%%%%%%%%%%%%%%%%%%%%%%%%%
\title{A Decidable Quantified Fragment of Set Theory Involving Ordered
Pairs with Applications to Description Logics}
\titlerunning{A Quantified Fragment of Set Theory Involving Ordered
Pairs} %optional, in case that the title is too long; the running title should fit into the top page column
\Copyright[nc-nd]%choose "nd" or "nc-nd"
          {John Q. Open and Joan R. Access}

\author{Domenico Cantone}
\author{Cristiano Longo}
\author{Marianna Nicolosi Asmundo}
\affil{Dipartimento di Matematica e Informatica, Universit\`a di Catania\\
  Viale Andrea Doria 6, 95125, Catania,  Italy\\
  \texttt{\{cantone|longo|nicolosi\}@dmi.unict.it}}

\Copyright[nc-nd]%choose "nd" or "nc-nd"
          {Domenico Cantone and Cristiano Longo and Marianna Nicolosi Asmundo}

% \subjclass{\textbf{F.4.1 Mathematical Logic (mechanical theorem proving, set theory)}}% mandatory: Please choose ACM 1998 classifications from http://www.acm.org/about/class/ccs98-html . E.g., cite as "F.1.1 Models of Computation".
\subjclass{\textbf{I.2.3, F.4.1, I.2.4}}% mandatory: Please choose ACM 1998 classifications from http://www.acm.org/about/class/ccs98-html . E.g., cite as "F.1.1 Models of Computation".
\keywords{\textbf{NP-complete decision procedures, set theory, description logic}}% mandatory: Please provide 1-5 keywords

%%%%%%%%%%%%%%%%%%%%%%%%%%%%%%%%%%%%%%%%%%%%%%%%%%%%%%%%%

%Editor-only macros (do not touch as author)%%%%%%%%%%%%%%%%%%%%%%%%%%%%%%%%%%%
\serieslogo{}%please provide filename (without suffix)
\volumeinfo%(easychair interface)
  {Billy Editor, Bill Editors}% editors
  {2}% number of editors: 1, 2, ....
  {Conference title on which this volume is based on}% event
  {1}% volume
  {1}% issue
  {1}% starting page number
\EventShortName{}
\DOI{10.4230/LIPIcs.xxx.yyy.p}% to be completed by the volume editor
%%%%%%%%%%%%%%%%%%%%%%%%%%%%%%%%%%%%%%%%%%%%%%%%%%%%%%%%%

%\usepackage{amssymb}
\usepackage{amsfonts}
%\usepackage{amsmath}
\usepackage{dl}
\usepackage{mls}
\usepackage{enumitem}
\usepackage{setspace}



\newcommand{\Lang}{\ensuremath{\mathbf{\forall_{0}^{\pi}}}\xspace}
\newcommand{\LangBounded}[1]{\ensuremath{(\forall_{0}^{\pi})^{\leq #1}}\xspace}
\newcommand{\nonpairs}[1]{\bar{\pi}(#1)}

\newcommand{\assignment}[1]{\ensuremath{M_{#1}}}
\newcommand{\pairf}[1]{\ensuremath{\pi_{#1}}}
\newcommand{\inter}{\ensuremath{\mathbf{I}}\xspace}
\newcommand{\atset}{\mathcal{S}}
\newcommand{\vheight}{\mathsf{height}_{\atset}}
\newcommand{\expansion}[2]{\mathsf{exp}_{#2}(#1)}
\newcommand{\Expansion}[2]{\mathsf{Exp}_{#2}(#1)}


%TOOLS
\newcommand{\memclosure}[1]{\in_{#1}^{+}}
\newcommand{\aslit}[1]{\mbox{``}#1\mbox{''}}
\newcommand{\real}{\mathbf{R}}

\newcommand{\mtrue}{\ensuremath{\mathbf{t}}}
\newcommand{\mfalse}{\ensuremath{\mathbf{f}}}

%DESCRIPTION LOGICS
\newcommand{\dlLang}{\ensuremath{\mathcal{DL\langle}\Lang\mathcal{\rangle}}\xspace}
\newcommand{\dlLangMeta}{\ensuremath{\mathcal{DLM\langle}\Lang\mathcal{\rangle}}\xspace}
\newcommand{\selfrestriction}[1]{\exists #1.\mathsf{Self}}
\newcommand{\istransitive}[1]{\mathsf{Trans}(#1)}
\newcommand{\isreflexive}[1]{\mathsf{Ref}(#1)}
\newcommand{\isantisymmetric}[1]{\mathsf{ASym}(#1)}
\newcommand{\symmetricclosure}[1]{\mathsf{sym}(#1)}
\newcommand{\roleidentity}[1]{\mathsf{id}(#1)}
\newcommand{\universalrole}{\mathbb{U}}

\begin{document}

\maketitle

\begin{abstract}
We present a decision procedure for a quantified fragment of
set theory, called \Lang, involving ordered pairs and some operators to manipulate them.
When our decision procedure is applied to \Lang-formulae whose quantifier prefixes
have length bounded by a fixed constant, it runs in nondeterministic
polynomial-time.

Related to the fragment \Lang, we also introduce a description logic,
\dlLang, which provides an unusually large set of constructs, such as,
for instance, Boolean constructs among roles.  The set-theoretic
nature of the description logics semantics yields a straightforward
reduction of the knowledge base consistency problem for \dlLang to the
satisfiability problem for \Lang-formulae with quantifier prefixes of
length at most 2, from which the \textsc{NP}-completeness of reasoning
in \dlLang follows.  Finally, we extend this reduction to cope with
\textsf{SWRL} rules.
\end{abstract}

\section{Introduction}\label{INTRO}


\emph{Computable Set Theory} is a research field, started around
thirty years ago, devoted to the study of the decision problem for
fragments of set theory (see \cite{CanFerOmo89a,CanOmoPol01} for a
thorough account of the state-of-the-art until 2001).  The most
efficient decision procedures devised in this context have been
implemented in the inferential core of the system
\textsf{{\AE}tnaNova/Referee}, described
in~\cite{CanOmoSchUrs03,OmoCanPolSch06,SchwCanOmoPol11}.


The first unquantified sublanguage of set theory that has been proved
decidable is \emph{Multi-Level Syllogistic} (in short \mls).  \mls
involves the set predicates $\in$, $\subseteq$, $=$, the Boolean set
operators $\cup$, $\cap$, $\setminus$, and the connectives of
propositional logic (cf.\ \cite{FerOmoSch1980}).  Subsequently,
several extensions of \mls with various combinations of operators
(such as singleton, powerset, unionset, etc.)  and predicates (on
finiteness, transitivity, etc.)  have been proved to have a solvable
satisfiability problem. Also, some extensions of \mls with various
map\footnote{According to \cite{SchDewSchDub1986}, we use the term
`maps' to denote sets of ordered pairs.}
constructs have been shown to be decidable (cf.\
\cite{CanSch91,CanLonNic2010}).


Concerning \emph{quantified} fragments, of particular interest to
us is the restricted quantified fragment of set theory $\forall_{0}$,
which has been proved to have a decidable satisfiability problem in
\cite{BreFerOmoSch1981}.  We recall that $\forall_{0}$-formulae are
propositional combinations of restricted quantified prenex formulae
$(\forall y_1 \in z_1) \cdots (\forall y_n \in z_n)p$, where $p$ is a
Boolean combination of atoms of the forms $x \in y$, $x=y$, and no
$z_j$ is a $y_i$ (i.e., nesting among quantified variables is not
allowed).  The same paper considered also the extension with another
sort of variables representing \emph{single-valued} maps, the map
domain operator, and terms of the form $f(t)$ (representing the value
of the map $f$ on a function-free term $t$).  However, neither
one-to-many, nor many-to-one, nor many-to-many relationships can be
represented in this language.  We observe that the
$\forall_{0}$-fragment is very close to the undecidability boundary,
as shown in \cite{ParPol93}. In fact, if nesting among quantified variables 
in prenex formulae of type $(\forall y_1 \in z_1) \cdots (\forall y_n
\in z_n)p$ are allowed and a predicate stating that a set is an 
unordered pair is also admitted, then it turns out that the 
satisfiability for the 
resulting collection of formulae is undecidable.

In this paper we present a decision procedure for the novel
fragment of set theory \Lang, which extends the fragment $\forall_{0}$
with ordered pairs and various constructs related to them, thus 
further thinning the gap between the decidable and the undecidable.
% 
A considerable amount of set-theoretic constructs can be expressed 
by \Lang-formulae, in particular constructs on \emph{multi-valued} maps
like map inverse, Boolean operator among maps, map transitivity, and
so on.  Furthermore, when restricted to formulae with quantifier
nesting bounded by a constant, our decision procedure runs in
nondeterministic polynomial-time.  This fragment has also interesting
applications in the field of \emph{knowledge representation}.

Applications of Computable Set Theory to knowledge representation have
been recently proposed in \cite{CanLonPis2010}, where the
correspondence between (decidable) fragments of set theory and
\emph{Description Logics} (a well established framework for knowledge
representation systems; see \cite{DLHANDBOOK} for a quite complete
overview) is exploited by introducing the very expressive description
logic $\dlmlsscart$.

Description logics are a family of logic based formalisms widely used
in knowledge representation. In particular, several results and
decision procedures devised in this context have been profitably
employed in the area of the Semantic Web (cf. \cite{HorKutSat2006}).
%
The key problem in description logic is to determine whether a
knowledge base $\mathcal{K}$ is \emph{consistent} (knowledge base
consistency is formally described in Section \ref{DL}), and many other
reasoning tasks can be reduced to it.  Unfortunately, this problem is
\textsc{ExpTime}-hard (cf.  \cite[Theorem 3.27, page 132]{DLHANDBOOK})
also for $\al$, a basic description logic with a very limited
expressive power.  However, \cite{CanLonPis2010} shows how a better
computational complexity can be achieved by imposing some limitations
on the usage of existential quantification and number restrictions
(definitions of these two constructs are reported in
Table~\ref{DLCONS}).

The quantified nature of the language \Lang and the pair-related
constructs it provides allow a straightforward mapping of numerous
description logic constructs to \Lang-formulae.  The resulting
description logic, called \dlLang, extends those presented in
\cite{CanLonPis2010} with several constructs like, for instance, role
transitivity, self restrictions, and role identity.  It also allows
\emph{finite} existential restrictions of the form $\exists R.\{a_1,
\ldots, a_n\}$ to be used without limitations.  Furthermore, it turns
out that the consistency problem for \dlLang-knowledge bases is
\textsc{NP}-complete.  This is a quite significant result since in
most of the cases in which Boolean operators among roles are present
the consistency problem turns out to be \textsc{NExpTime}-hard
(cf.~\cite{Lutz2001}).


Finally, we observe that \textsf{SWRL} rules (cf.  \cite{HorPat2004}) can be
easily embedded in \dlLang without disrupting decidability.  \textsf{SWRL}
rules
are a simple form of Horn-style rules, which were proposed with the
aim of increasing the expressive power of description logics.
Here we consider only a restricted set of \textsf{SWRL} rules,
namely those which do not contain \emph{data literals}. Extending
description logics with \textsf{SWRL} rules in general leads to
undecidability.  In \cite{MotSatStu2005} this issue has been overcome
by restricting the applicability of rules to a finite set of named
individuals.
%
Another approach, studied in \cite{KroRudHit2008}, consists in
restricting to rules which can be \emph{internalized}, i.e. rules
which can be converted into knowledge base statements.

The paper is organized as follows.  Section~\ref{LANG} presents the
precise syntax and semantics of the language \Lang.  A decision
procedure for \Lang is then developed in Section~\ref{DECPROC}.  In
Section~\ref{DL} the correspondences of \Lang with description logics
are exploited by introducing the novel description logic \dlLang,
whose extension with \textsf{SWRL} rules is studied in Section~\ref{SWRL}.
Finally, concluding remarks and some hints to future work are given in
Section~\ref{CONCLUSIONS}.

\section{The language \Lang}\label{LANG}
The language \Lang is a quantified fragment of set theory which
contains a denumerable infinity of \emph{variables}, $\Vars = \{x, y,
z, \ldots \}$, the binary \emph{pairing} operator $[\cdot,\cdot]$, the
monadic function $\nonpairs{\cdot}$, which represents the non-pair
members of a set, the relators $\in, =$, the Boolean connectives of
propositional logic $\neg, \wedge, \vee, \rightarrow,
\leftrightarrow$, parentheses, and the universal quantifier
$\forall$.

A \emph{quantifier-free \Lang-formula} is any propositional
combination of \emph{atomic \Lang-formulae}.  These are expressions of
the following types:
\begin{equation}\label{ATOMICFORMULAE}
    x \in \nonpairs{z},\quad [x,y] \in z,\quad x=y,
\end{equation}
with $x,y,z \in \Vars$.  Intuitively, terms of the form $[x,y]$
represent ordered pairs of sets.

A \emph{simple prenex \Lang-formula} is a formula $Q_1\cdots Q_n
\varphi$, with $n \geq 0$, where $\varphi$ is a quantifier-free
\Lang-formula, each $Q_i$ is a restricted universal quantifier of form
$(\forall x \in \nonpairs{y})$ or of the form $(\forall [x,x'] \in y)$
(we will refer to $x$ and $x'$ as \emph{quantified variables} and to
$y$ as \emph{domain variable}), and no variable can occur both as a
quantified and a domain variable, i.e., roughly speaking, no $x$ can
be a $y$.

Finally, a \Lang-\emph{formula} is any finite conjunction of simple
prenex \Lang-formulae.

Semantics of the \Lang-language is based upon the von Neumann
standard cumulative hierarchy $\VNU$ of sets, which is
defined as follows:
\[
\begin{array}{rcll}
  \VNU_0          & = & \emptyset
\\
  \VNU_{\gamma+1} & = & \mathcal{P}(\VNU_\gamma) \,,
  & \textrm{for each ordinal $\gamma$}
\\
  \VNU_\lambda    & = & \bigcup_{\mu < \lambda} \VNU_\mu
  \,,& \textrm{for each limit ordinal $\lambda$}
\\
  \VNU & = & \bigcup_{\gamma \in \mathit{On}} \VNU_{\gamma} \,,
\end{array}
\]
where $\mathcal{P}(\cdot)$ is the powerset operator and
$\mathit{On}$ denotes the class of all ordinals.

A \Lang-\emph{interpretation} is a pair $\inter=(\assignment{\inter},
\pairf{\inter})$, where $\assignment{\inter}$ is a total function
which maps each variable into a set of \VNU, and $\pairf{\inter}$ is a
\emph{pairing function} over sets. We recall that a
\emph{pairing function} $\pi$ is a binary operation over sets such
that $\pi(u,v)=\pi(u',v') \iff u=u' \wedge v=v'$ and the class $u
\times_{\pi} v \defAs \{ \pi(s,t) : s \in u \wedge t \in v\}$ is a set
of \VNU, for all $u,v,u',v' \in \VNU$. 

Let $W$ be a finite subset of
$\Vars$, we say that $\inter' = (\assignment{\inter'},
\pairf{\inter})$ is a $W$-\emph{variant} of $\inter$ if $M_{\inter'} y
= M_{\inter} y$, for $y \in \Vars \setminus W$.  To any term of the
form $x$, $[x,y]$, and $\nonpairs{x}$, a \Lang-interpretation \inter
associates a set in \VNU as follows:
\[
\begin{array}{rcl}
 \inter x & \defAs & \assignment{\inter} x\\
 \inter [x,y] & \defAs & \pairf{\inter} (\inter x, \inter y)\\
 \inter \nonpairs{x} & \defAs &
 \inter x \setminus \{\pi_{\inter}(u,v) : u, v \in \VNU\},
\end{array}
\]
for all $x,y \in \Vars$.

A \Lang-interpretation evaluates atomic \Lang-formulae to the truth
values \mtrue\ (true) and \mfalse\ (false) in the usual way, by
interpreting `$\in$' and `$=$' as the membership and the equality
relations between sets, respectively.
Evaluation of quantifier-free
\Lang-formulae is carried out according to the standard rules of
propositional logic, and simple prenex \Lang-formulae are evaluated as
follows:
\begin{itemize}
\item $\inter(\forall x \in \nonpairs{y})\varphi = \mtrue$ iff
$\inter' \varphi = \mtrue$ for every $\{x\}$-variant $\inter'$ of
$\inter$ such that $\inter' x \in \inter' \nonpairs{y}$,

\item $\inter(\forall [x,y] \in z)\varphi = \mtrue$ iff $\inter'
\varphi = \mtrue$ for every $\{x,y\}$-variant $\inter'$ of $\inter$
such that $\inter' [x,y] \in \inter' z$.
\end{itemize}

A \Lang-interpretation \inter which evaluates a \Lang-formula
$\varphi$ to true is said to be a \emph{model} for $\varphi$ (and we
write $\inter \models \varphi$).  A \Lang-formula is said to be
\emph{satisfiable} if it admits a model.  Thus the satisfiability
problem (in short, s.p.) for \Lang-formulae consists in determining
whether a \Lang-formula is satisfiable or not.
%
Observe that in the context of satisfiability, all free variables in a
\Lang-formula may be regarded as existentially quantified.

In the following section we present a decision procedure for the
s.p.\ for \Lang-formulae.

\section{A decision procedure for \Lang}\label{DECPROC}
In this section we solve the s.p.\ for \Lang-formulae.  
In particular,
we will prove that a \Lang-formula is satisfiable if and only if
there exists a finite collection of atomic \Lang-formulae which
\emph{represents} a model for the formula.  
We begin by
introducing the notions of skeletal representations and of their
realizations: these are, respectively, collections of atomic
\Lang-formulae with an acyclic membership relation among their
variables, and suitably defined \Lang-interpretations.  In particular,
we will focus on skeletal representations ``completed'' w.r.t. the
predicate ``='' over a set of variables $V$, which we call
$V$-extensional.
It turns out, as will be shown in Lemma \ref{REALIZATIONLEMMA}, that
each $V$-extensional skeletal representation is modeled correctly by
any realization associated with it.  Finally, we prove the main result
of this section, namely that a \Lang-formula $\varphi$ with free
variables $V$ is satisfiable if and only if it is satisfied by the
realization associated with a suitable $V$-extensional skeletal
representation whose size is bounded by the cardinality of $V$ (cf.\
Theorem \ref{SAT}).  The latter result entails immediately the
decidability of the fragment \Lang of our interest.

Given a \Lang-formula $\varphi$, we denote with $\varphi^{x}_{y}$ the
formula obtained by replacing each free occurrence of $x$ in $\varphi$
with $y$ and with $\Vars(\varphi)$ the collection of the free
variables of $\varphi$.  Likewise, given a finite collection $\atset$
of atomic \Lang-formulae, we denote with $\Vars(\atset)$ the
collection of the variables occurring in the formulae of $\atset$.  In
addition, we indicate with $\memclosure{\atset}$ (the \emph{membership
closure} of $\atset$) the minimal transitive relation on
$\Vars(\atset)$ such that the following conditions hold:
\begin{itemize}
    \item if $\aslit{x \in \nonpairs{z}} \in \atset$, then $x
    \memclosure{\atset} z$;

    \item if $\aslit{[x,y] \in z} \in \atset$, then $x
    \memclosure{\atset} z \wedge y \memclosure{\atset} z$.
\end{itemize}

A collection $\atset$ of atomic \Lang-formulae is a
\emph{skeletal representation} if $x \not \memclosure{\atset} x$, for all
$x \in \Vars(\atset)$.

Let $\atset$ be a skeletal representation.  We define the
\emph{height} of a variable $x \in \Vars(\atset)$ with respect to
$\atset$ (which we write $\vheight(x)$) as the length $n$ of the
longest $\memclosure{\atset}$-chain of the form $x_1
\memclosure{\atset} \ldots \memclosure{\atset} x_n \memclosure{\atset}
x$ ending at $x$, with $x_1, \ldots ,x_n \in \Vars(\atset)$.  Thus,
$\vheight(x)=0$ if $y \not\memclosure{\atset} x$, for all $y \in
\Vars(\atset)$.

A skeletal representation $\atset$ is said to be
$V$\emph{-extensional}, for a given set of variables $V$, if the following conditions hold:
\begin{itemize}
 \item if $\aslit{x=y} \in \atset$, then $x,y \in V$ and
 $\alpha_y^x$ and $\alpha_x^y$ belong to $\atset$, for each atomic
 formula $\alpha$ in $\atset$;

 \item if $\aslit{x=y} \notin \atset$, for some $x,y \in V$, then the
 variables $x$ and $y$ must be explicitly \emph{distinguished} in
 $\atset$ either by some variable $z$, in the sense that $\aslit{z \in
 \nonpairs{x}} \in \atset$ iff $\aslit{z \in \nonpairs{y}} \notin
 \atset$, or by some pair $[z,z']$, in the sense that $\aslit{[z,z']
 \in x} \in \atset$ iff $\aslit{[z,z'] \in y} \notin \atset$.
\end{itemize}


Skeletal representations allow one to define special
\Lang-interpretations, called \emph{realizations}, which were first
introduced in \cite{CanFer1995}, though with a slightly different
meaning.  To this purpose we introduce the following family
$\{\pi_n\}_{n \in \mathbb{N}}$ of pairing functions, recursively
defined by
$$
\begin{array}{rcl}
  \pi_0(u,v) & \defAs & \{u, \{u,v\}\}\\
  \pi_{n+1}(u,v) & \defAs & \{ \pi_n(u,v)\}\, ,
\end{array}
$$
for every $u,v \in \VNU$.


\begin{definition}[Realization]\label{REALIZATION}
Let $\atset$ be a skeletal representation, let $V$ and $T$ be two
finite, nonempty, and disjoint sets of variables such that
$\Vars(\atset) \subseteq V \cup T$, and let $\sigma$ be a bijection
from $T$ onto $\{1,2,\ldots,|T|\}$.  We extend the function
$\vheight(\cdot)$ also to variables $x \in (V \cup T) \setminus
\Vars(\atset)$ by putting for such variables $\vheight(x) \defAs 0$.

Then the \emph{realization} of $\atset$ relative to $(V,T)$ is
the \Lang-interpretation $\real=(\assignment{\real}, \pairf{\real})$
such that $\pairf{\real} \defAs \pi_{|V|+|T|}$ and, recursively on
$\vheight(x)$ for $x \in V \cup T$,
\[
\assignment{\real} x \defAs \left\{ \real y : \aslit{y \in
\nonpairs{x}} \in \atset\right\} \cup \left\{ \real [y,z] :
\aslit{[y,z] \in x} \in \atset\right\} \cup s(x)\, ,
\]
where
\[
s(x) \defAs  \begin{cases}
\left\{ \left\{k+1, k, \sigma(x)\right\} \right\} & \mbox{ if } x \in T\\
\emptyset &\mbox{ otherwise, }\\
\end{cases}
\]
with $k=|V| \cdot (|V| + |T| + 3)$.\footnote{We are assuming that
integers are represented \emph{\`a la} von Neumann, namely $0 \defAs
\emptyset$ and, recursively, $n+1 \defAs n \cup \{n\}$.} \qed
\end{definition}

Realizations have useful properties, stated by the following
lemma.
\begin{lemma}\label{REALIZATIONLEMMA}
Let $\atset$, $V$, $T$, $\sigma$, and $k$ be as in Definition
\ref{REALIZATION} and let $\real$ be the realization of $\atset$
relative to $(V,T)$.  If $\atset$ is $V$-extensional, then for every
$x,y,z \in V \cup T$ the following conditions hold:
\begin{enumerate}[label=\textbf{(R\arabic*)},leftmargin=28pt]
  \item\label{R1} $\real x \neq \pairf{\real}(u,v)$ for $u,v \in
  \VNU$;

  \item\label{R2} $\real x \neq \{k+1, k, i\}$ for
  $1 \leq i \leq |T|$;

  \item\label{R3} $\real x = \real y$ iff either $\aslit{x=y}
  \in \atset$ or $x$ and $y$ coincide;

  \item\label{R4} $\real x \in \real \nonpairs{y}$ iff $\aslit{x \in
  \nonpairs{y}} \in \atset$;

  \item\label{R5} $\real [x,y] \in \real z$ iff $\aslit{[x,y] \in z}
  \in \atset$.
\end{enumerate}
\end{lemma}

\begin{proof}
To prove \ref{R1}, we establish the more general property
\begin{equation}\label{REALIZATION2EQ2}
\text{if~ } \vheight(x) \leq n \leq |V|+|T| ,
\text{ ~then~ }
\real x  \neq \pi_n(u, v), \text{ for } x \in V \cup T
\text{ and } u, v \in \VNU.
\end{equation}
Let $n \leq |V|+|T|$ and let us assume by way of contradiction that
$\real x =\pi_n(u,v)$ for some $u,v \in \VNU$ and some $x \in V \cup
T$ of minimal height such that $0 \leq \vheight(x) \leq n$.

We can rule out at once the case in which $n=0$, as in this case
$\vheight(x) = 0$, so that $|\real x| \leq 1$, and therefore $\real x
\neq \pi_0(u,v)$, since $|\pi_0(u,v)| = 2$.

Thus, we can assume that $n>0$.  Let us consider first the case in
which $\vheight(x) = 0$.  If $x \in V$ then, by the very definition of
realization, we have $\real x = \emptyset \neq \pi_n(u,v)$.  On the
other hand, if $x \in T$, then $\real x = \{\{k+1,k,\sigma(x)\}\}$ and
since $|\{k+1,k,\sigma(x)\}| > |\pi_{n-1}(u,v)|$ and $\pi_n(u,v) =
\{\pi_{n-1}(u,v)\}$, it follows that $\real x \neq \pi_n(u,v)$.  In
both cases we found a contradiction, so that we must have $\vheight(x)
> 0$.

On the other hand, if $\vheight(x)>0$, our absurd hypothesis $\real x
= \pi_{n}(u,v) =\{ \pi_{n-1}(u,v) \}$ and the definition of realization imply that
either
\begin{enumerate}[label=\textbf{(\roman*)},leftmargin=23pt]
\item\label{REALIZATIONLEMMA1} $\pi_{n-1}(u,v)=\{k+1,k,\sigma(x)\}$,
but provided that $x \in T$, or

\item\label{REALIZATIONLEMMA2} $\pi_{n-1}(u,v)=\real y$,
for some $y$ such that $\aslit{y \in \nonpairs{x}} \in \atset$, or

\item\label{REALIZATIONLEMMA3} $\pi_{n-1}(u,v)=\real [y,z] =
\pi_{|V|+|T|} (\real y, \real z)$, for some $y,z$ such that
$\aslit{[y,z] \in x} \in \atset$.
\end{enumerate}
We can exclude at once case \ref{REALIZATIONLEMMA1}, since
$|\pi_{n-1}(u,v)|\leq 2 < |\{k+1,k,\sigma(x)\}|$.  Case
\ref{REALIZATIONLEMMA2} can be excluded as well, since it would
contradict the minimality of $\vheight(x)$, as $\vheight(y) <
\vheight(x)$.  In case \ref{REALIZATIONLEMMA3}, from elementary
properties of our pairing functions $\pi_{i}$ it would follow that
$|V|+|T|= n-1$, contradicting our initial assumption that $n \leq
|V|+|T|$.  Thus (\ref{REALIZATION2EQ2}) holds.

In view of (\ref{REALIZATION2EQ2}), to establish \ref{R1} it is now
enough to observe that $\vheight(x) < |V|+|T|$.

\smallskip

Next, since $\rank(\{k+1, k, i\}) = k+2$, for $1 \leq i \leq |T|$ (as
$k > |T|$),\footnote{We recall that the \emph{rank} of a set $u \in
\VNU$ denotes the least ordinal $\gamma$ such that $u \subseteq
\VNU_{\gamma}$ (i.e., $u \in \VNU_{\gamma+1}$).} to establish \ref{R2}
it will be enough to show that $\rank(\real x) \neq k+2$, for $x \in V
\cup T$.  Thus, let $x \in V \cup T$.  If $y \memclosure{\atset} x$,
for some $y \in T$, then $\rank(\real x) \geq \rank(\real y) \geq k+3$.
The same conclusion can be reached also in the case in which $x \in
T$.  On the other hand, if $y \not\memclosure{\atset} x$, for any $y
\in T$, and $x \in V$, it can easily be proved by induction on
$\vheight(x)$ that $\rank(\real x) \leq (|V| + |T| + 3)\cdot \vheight(x)
\leq |V|\cdot (|V| + |T|+3) = k$.  Hence, in any case $\rank(\real x)
\neq k+2$ holds, proving \ref{R2}.

\smallskip

Concerning \ref{R3}, we observe preliminarily that if $\aslit{x=y} \in
\atset$, then $\real x = \real y$ is a direct consequence of the
$V$-extensionality of $\atset$.  Thus, to complete the proof of
\ref{R3} it is enough to show that if $\real x = \real y$, for
distinct variables $x,y \in V \cup T$, then $\aslit{x=y} \in \atset$.
So, assume that $\aslit{x=y} \notin \atset$, for two distinct
variables $x,y \in V \cup T$ and consider first the case in which
either $x$ or $y$, say $y$, is a variable in $T$.  From the definition
of realization it follows that $\{k+1,k,\sigma(y)\} \in \real y$,
while from \ref{R2} and the fact that $\{k+1,k,\sigma(y)\}$ is not a
pair with respect to $\pi_{|V|+|T|}$, it follows that
$\{k+1,k,\sigma(y)\} \notin \real x$, unless $x \in T$ and
$\{k+1,k,\sigma(y)\} = \{k+1,k,\sigma(x)\}$.  But in such a case, we
would have $\sigma(x) = \sigma(y)$ and therefore $x$ and $y$ must
coincide, contradicting our initial assumption that $x$ and $y$ are
distinct variables. Therefore we have $\real x \neq \real y$.

Next, let us assume that $x,y \in V$. We will induction on
$\max(\vheight(x), \vheight(y))$. From the $V$-extensionality of
$\atset$ it follows that $x,y$ are distinguished in $\atset$ by a
variable $z$ or by a pair $[z',z'']$.  Let us first assume that $x,y$
are distinguished in $\atset$ by a variable $z$.  If $\aslit{z \in
\nonpairs{x}} \in \atset$ and $\aslit{z \in \nonpairs{y}} \notin
\atset$, then for all $w$ such that $\aslit{w \in \nonpairs{y}} \in
\atset$ we have $\real z\neq \real w$ by the inductive hypothesis,
since $\vheight(z) < \vheight(x)$ and $\vheight(w)<\vheight(y)$.
Furthermore, from \ref{R1} it follows also that $\real z \neq \real
[w,w']$, for all $w,w'$ such that $\aslit{[w,w'] \in y} \in \atset$.
Thus $\real z \in \real x \setminus \real y$.  If $\aslit{z \in
\nonpairs{y}} \in \atset$ and $\aslit{z \in \nonpairs{x}} \notin
\atset$ we can prove that $\real z \in \real y \setminus \real x$ in
an analogous way. In both case we have $\real x \neq \real y$.
%
On the other hand, if $x,y$ are distinguished by a pair $[z',z'']$, we
can argue as follows.  Assume first that $\aslit{[z',z''] \in x} \in
\atset$ and $\aslit{[z',z''] \in y} \notin \atset$.  Plainly, $\real
[z',z''] \in \real x$.  If $\real [z',z''] \in \real y$, then by
\ref{R1} $\real [z',z''] = \real [w',w'']$, for a pair $[w',w'']$ such
that $\aslit{[w',w''] \in y} \in \atset$.  Since $\pi_{|V|+|T|}$ is a
pairing function, we have $\real z'=\real w'$ and $\real z''=\real
w''$. Considering that
$\vheight(z'),\vheight(z'') < \vheight(x)$ and that
$\vheight(w'),\vheight(w'')<\vheight(y)$,
the inductive
hypothesis yields that
\begin{itemize}
    \item $z'$ and $w'$ coincide or $\aslit{z' = w'}$ is in
    $\atset$, and

    \item $z''$ and $w''$ coincide or $\aslit{z'' = w''}$ is in
    $\atset$.
\end{itemize}
But then, by the $V$-extensionality of $\atset$, $\aslit{[z',z''] \in
y}$ would be in $\atset$, a contradiction.  Hence, $\real [z',z''] \in
\real x \setminus \real y$.  Analogously, if $\aslit{[z',z''] \in x}
\notin \atset$ and $\aslit{[z',z''] \in y} \in \atset$, we have $\real
[z',z''] \in \real y \setminus \real x$.  Therefore, in both cases we
have $\real x \neq \real y$, proving \ref{R3}.

The cases \ref{R4} and \ref{R5} are easy consequences of \ref{R1},
\ref{R2}, and \ref{R3}.  Details are left to the reader. This
completes the proof of the lemma.
\end{proof}

Realizations act as \emph{minimal models} for skeletal
representations, in the sense that if $V,T$ are two disjoint sets of
variables, $\atset$ is a $V$-extensional skeletal representation such
that $\Vars(\atset) \subseteq V \cup T$, and $\real$ is the
realization of $\atset$ relative to $(V,T)$ (and to a bijection
$\sigma$), then $\real \models \alpha$ if and only if $\alpha \in
\atset$.

In the next theorem we show how skeletal representations can be
used to witness the satisfiability of \Lang-formulae.

\begin{theorem}\label{SAT}
Let $\varphi$ be a \Lang-formula, and let $V=\Vars(\varphi)$. Then
$\varphi$
is satisfiable iff there exists a $V$-extensional
skeletal representation $\atset$ such that:
\begin{enumerate}[label=(\roman*),leftmargin=18pt]
 \item\label{SAT1} $\Vars(\atset) \subseteq V \cup T$, for some $T$
 such that $1 \leq |T| < 2|V|$;

 \item\label{SAT3} $\real \models \varphi$, where $\real$ is the
 realization of $\atset$ relative to $(V,T)$.
\end{enumerate}

\end{theorem}
\begin{proof}
To prove the theorem, it is enough to exhibit a skeletal
representation $\atset$ that satisfies conditions \ref{SAT1} and
\ref{SAT3} above, given a model $\inter$ for $\varphi$.

Thus, let $\inter$ be a model for $\varphi$ and let $\Sigma = \{
\inter x : x \in V\}$.  As shown in \cite{CanFer1995}, there exists a
collection $\Sigma_0$ of size strictly less than $|\Sigma|$ which
witnesses all the inequalities among the members of $\Sigma$, in the
sense that $s \cap \Sigma_0 \neq s' \cap \Sigma_0$ for any two
distinct $s,s' \in \Sigma$.  Let us \emph{split} the pairs present in
$\Sigma_0$ (relative to the pairing function $\pairf{\inter}$ of
$\inter$) forming the collection
\[
 \Sigma_1 \defAs \{ s : s \in \Sigma_0 \wedge (\forall u,v \in
 \VNU)(s \neq \pairf{\inter}(u,v))\}\; \cup \;
 \bigcup \left\{ \{u,v\} : \pairf{\inter}(u,v) \in \Sigma_0 \right\}.
\]
Then we put
\[
 \Sigma_{2} \defAs \begin{cases}
 \Sigma_{1} \setminus \Sigma & \text{if } \Sigma_{1} \setminus \Sigma
 \neq \emptyset\\
%
 \{\emptyset\} & \text{otherwise}
 \end{cases}
\]
and let $T$ be any collection of variables in $\Vars$, not already
occurring in $\varphi$, such that $|T|=|\Sigma_2|$ (so that $|T| \geq
1$).  Notice that $|T| \leq 2|\Sigma_0| + 1 < 2|V|$.

Finally, we define our skeletal representation as the collection
$\atset$ of atomic \Lang-formulae such that:
\[
 \begin{array}{rcl}
  \aslit{x \in \nonpairs{y}} \in \atset& \iff & \inter x \in \inter \nonpairs{y}\\
  \aslit{[x,y] \in z} \in \atset & \iff & \inter [x,y] \in \inter z\\
  \aslit{x=y} \in \atset & \iff & \inter x = \inter y \text{ ~and~ }
  x,y \in V
 \end{array}
\]
for all $x,y,z \in V \cup T$.

As can be easily verified, the above construction process yields a
$V$-extensional skeletal representation $\atset$ satisfying condition
\ref{SAT1} of the theorem.

We prove next that also condition \ref{SAT3} is satisfied, i.e. $\real
\models \varphi$ holds, where $\real$ is the realization of $\atset$
relative to $(V,T)$.  This amounts to showing that $\real$ models
correctly all conjuncts of $\varphi$.  These are simple prenex
\Lang-formulae whose free variables belong to $V \cup T$ and whose
domain variables belong to $V$, which are correctly modeled by
$\inter$.  It will therefore be enough to prove the following general
property stating that
\begin{equation}\label{SATEQ1}
\inter \models \psi  \Longrightarrow \real \models \psi,
\end{equation}
for every simple prenex \Lang-formula $\psi$ such that $\Vars(\psi)
\subseteq V \cup T$ and whose domain variables, if any, belong to $V$.

We prove (\ref{SATEQ1}) by induction on the length of the quantifier prefix
of $\psi$.

When $\psi$ is quantifier-free, (\ref{SATEQ1}) follows from
propositional logic, by observing that the definition of $\atset$
together with conditions \ref{R3}, \ref{R4}, and \ref{R5} of Lemma
\ref{REALIZATIONLEMMA} yield that $\inter \alpha = \real \alpha$, for
each atomic \Lang-formula $\alpha$ such that $\Vars(\alpha) \subseteq
V \cup T$.

For the inductive step, let $\psi$ have either the form $(\forall x
\in \nonpairs{y})\chi$ or the form $(\forall [x,y] \in z)\chi$, with
$\chi$ a simple prenex \Lang-formula having one less quantifier than
$\psi$.  For the sake of simplicity, we consider here only the case in
which $\psi$ has the form $(\forall x \in \nonpairs{y})\chi$, as the
other case can be dealt with much in the same manner.  We remark that,
by hypothesis, the domain variable $y$ in $(\forall x \in
\nonpairs{y})\chi$ belongs to $V$.

Let us assume that $\inter \models \psi$.  To complete the inductive
proof of (\ref{SATEQ1}) we need to show that $\real \models \psi$.  From $\inter \models
\psi$ it follows that $\inter \models (w \in \nonpairs{y}) \rightarrow
\chi_w^x$, for every variable $w$, and in particular for every
variable $w \in W$, where
$
 W \defAs \{ w \in V \cup T : \aslit{w \in \nonpairs{y}} \in \atset\}.
$
Let $w \in W$.  We clearly have $\inter \models w \in \nonpairs{y}$,
and therefore $\inter \models \chi_w^x$.  Plainly, $\Vars(\chi_w^x)
\subseteq V \cup T$.  In addition, all domain variables in $\chi_w^x$
belong to $V$, since this is the case for all domain variables in
$\chi$ and $w$ can not appear in $\chi_w^x$ as a domain variable,
since $x$ is a quantified variable of $\psi$ and as such can not
appear also as a domain variable in $\psi$, and therefore in $\chi$.
Hence, by inductive hypothesis, we have $\real \models \chi_w^x$ and,
\emph{a fortiori}, $\real \models (w \in \nonpairs{y}) \rightarrow
\chi_w^x$.

Notice that the latter relation holds also for $w \in (V \cup T)
\setminus W$, since in this case $\inter \not\models (w \in
\nonpairs{y})$ and therefore, as observed above, $\real \not\models (w
\in \nonpairs{y})$.  Thus we have
\begin{equation}
    \label{eq_allW}
    \real \models (w \in \nonpairs{y}) \rightarrow \chi_w^x,
\end{equation}
for every $w \in V \cup T$.  We show that (\ref{eq_allW}) implies
$\real \models (\forall x \in \nonpairs{y})\chi$, which is what we
want to prove.

Indeed, if by contradiction $\real \not\models (\forall x \in
\nonpairs{y})\chi$, then $\real' \not\models (x \in \nonpairs{y})
\rightarrow \chi$, for some $\{x\}$-variant $\real'$ of $\real$, so
that $\real' \models (x \in \nonpairs{y})$ and $\real' \not\models
\chi$. But then
\[
\real' x \in \real' \nonpairs{y} = \real \nonpairs{y} \subseteq
\{\real z : \aslit{z \in \nonpairs{y}} \in \atset\}.
\]
Therefore $\real' x = \real z_{0}$, for some variable $z_{0}$ (in $V
\cup T$) such that the literal $\aslit{z_{0} \in \nonpairs{y}}$
belongs to $\atset$. Thus we have
$
\real \models z_{0} \in \nonpairs{y} \quad \text{and} \quad
\real \not\models (z_{0} \in \nonpairs{y}) \rightarrow
\chi^{x}_{z_{0}},
$
contradicting (\ref{eq_allW}).
%
Hence, $\real \models (\forall x \in \nonpairs{y})\chi$ holds,
completing the inductive proof of (\ref{SATEQ1}) and, in turn, the
proof of condition \ref{SAT3} of the theorem.
%
\end{proof}

Theorem \ref{SAT} yields a decision test for the s.p.\ for \Lang-formulae, as the number of possible $V$-extensional
skeletal representations satisfying condition \ref{SAT1} of the
theorem is finite, for any given \Lang-formula, and condition
\ref{SAT3} is effectively verifiable.  In the following section, we
analyze the s.p.\ for \Lang-formulae from a
complexity point of view.

\subsection{Complexity issues}

The s.p.\ for propositional logic can be easily
reduced to the one for \Lang-formulae as follows.  Given a
propositional formula $Q$, we construct in linear time a
quantifier-free \Lang-formula $\varphi_{Q}$, by replacing each
propositional variable $p$ in $Q$ with a corresponding atomic
\Lang-formula $x_{p} \in \nonpairs{U}$, where $U$ is a set variable
distinct from all set variables $x_{p}$ so introduced.  It is then
immediate to check that $Q$ is propositionally satisfiable if and only
if the resulting \Lang-formula $\varphi_{Q}$ is satisfiable.  Thus the
\textsc{NP}-hardness of the satisfiability of \Lang-formulae follows
immediately.

Having shown a lower bound for the s.p.\ for \Lang-formulae, we next
give an upper bound for it, proving that it is in the
\textsc{NExpTime} class and, furthermore, when restricted to a certain
useful collection of \Lang-formulae, it is \textsc{NP}-complete.

As proved in Theorem \ref{SAT}, satisfiability of a \Lang-formula
$\varphi$ can be tested by first guessing a skeletal representation
$\atset$, whose size is polynomial in the size $|\varphi|$ of
$\varphi$ (since $|\Vars(\atset)| < 3\cdot|\Vars(\varphi)|$), and then
verify that the formula $\varphi$ is modeled correctly by the
realization $\real$ of $S$ relative to $(V,T)$, where $V =
\Vars(\varphi)$ and $T = \Vars(\atset) \setminus \Vars(\varphi)$.
Construction of the realization $\real$ takes polynomial time, however
to verify that $\real \models \varphi$ can take exponential time.
Indeed, it is easy to check that $\real$ models correctly $\varphi$ if
and only if it satisfies the \emph{expansion}
$\Expansion{\varphi}{\atset}$ of $\varphi$ relative to $\atset$,
which we define shortly. For a simple prenex \Lang-formula $\psi$, we
put
\[
 \expansion{\psi}{\atset} \defAs \begin{cases}
          \hspace{22pt}\psi & \text{if }\psi \text{ is quantifier-free},\\
% 	
	  \bigwedge\limits_{\aslit{x' \in \nonpairs{y}} \in \atset}
	  \expansion{\chi_{x'}^x}{\atset} & \text{if
	  }\psi=(\forall x \in \nonpairs{y})\chi,\\
% 	
	  \bigwedge\limits_{\aslit{[x',y'] \in z} \in \atset}
	  \expansion{\chi_{x',y'}^{x,\,\,y}}{\atset}~~~ & \text{if }
	  \psi=(\forall [x,y] \in z)\chi.
 \end{cases}
\]
Then we put
\[
\Expansion{\varphi}{\atset} \defAs
\expansion{\varphi_1}{\atset}\wedge
\ldots \wedge \expansion{\varphi_n}{\atset},
\]
where $\varphi_{1}, \ldots, \varphi_{n}$ are the (simple prenex)
conjuncts of $\varphi$. If $\ell$ is the longest quantifier prefix of
the formulae $\varphi_{1}, \ldots, \varphi_{n}$, then it turns out
that $|\Expansion{\varphi}{\atset}| =
\mathcal{O}(|\varphi|^{2\ell}) = \mathcal{O}(|\varphi|^{2\cdot|\varphi|})$, and therefore to test whether $\real
\models \Expansion{\varphi}{\atset}$ takes at most exponential time,
showing that the s.p.\ for \Lang-formula is in \textsc{NExpTime}.

However, the same proof shows that if we restrict to the collection of
\Lang-formulae whose quantifier prefixes are bounded by a constant $h
\geq 0$, which we call $\LangBounded{h}$, then
$|\Expansion{\varphi}{\atset}|$ is only polynomial in $|\varphi|$, for
any $\LangBounded{h}$-formula $\varphi$, and therefore to test whether
$\real$ models correctly $\Expansion{\varphi}{\atset}$, and in turn to
test whether $\real \models \varphi$, takes polynomial time in
$|\varphi|$, proving the following result:
\begin{corollary}\label{UDECSIZEBOUNDED}
The s.p.\ for $\LangBounded{h}$-formulae is \textsc{NP}-complete, for
any $h \geq 0$.
\qed
\end{corollary}

In the rest of the paper we describe some applications of
\Lang-formulae in the field of knowledge representation.  More
specifically, in the next section we introduce a novel description
logic whose consistency problem can be reduced to the s.p.\ for
$\LangBounded{2}$-formulae.  Such description logic will then be
extended with Horn-style rules in Section~\ref{SWRL}.


\section{The description logic \dlLang}\label{DL}

Description logics are a family of logic-based formalisms which allow
to represent knowledge about a domain of interest in terms of
\emph{concepts} (which denote sets of elements), \emph{roles} (which
represent relations between elements), and \emph{individuals} (which
denote domain elements).  Each language in this family is mainly
characterized by its set of \emph{constructors}, which allow to form
complex terms starting from \emph{concept names}, \emph{role names},
and \emph{individual names} (see Table \ref{DLCONS} for the syntax and
semantics of the most widely used description logic constructs).  A
description logic \emph{knowledge base} is a finite set of statements
which define constraints on the domain structure.

Description logic semantics\footnote{Here we are recalling the
\emph{descriptive} semantics.  There are several other semantics that
are out of the scope of this paper.} is given in terms of
\emph{interpretations}.  An interpretation $\I$ consists of a nonempty
\emph{domain} $\Delta^{\I}$ and an interpretation function assigning
to each concept name a subset of $\Delta^{\I}$, to every role name a
relation over $\Delta^{\I}$, and to every individual name a domain
item in $\Delta^{\I}$.  An interpretation $\I$ extends recursively to
complex terms.  An interpretation \I that satisfies all the
constraints of a knowledge base $\mathcal{K}$ is said to be a
\emph{model} for $\mathcal{K}$.  A knowledge base is said to be
\emph{consistent} if it admits a model.  Thus the \emph{consistency
problem} for description logic knowledge bases is to determine whether
a knowledge base is consistent or not.

\begin{table}
\begin{math}
\begin{array}{|rclr|}
\hline
&&&\\
A^{\I} & \subseteq & \Delta^{\I} & \mbox{ (concept name)}\\
P^{\I} & \subseteq & \Delta^{\I} \times \Delta^{\I} & \mbox{ (role name)}\\
a^{\I} & \in & \Delta^{\I} & \mbox{ (individual name)}\\
&&&\\
\hline
&&&\\
\top^{\I}          & = & \Delta^{\I} & \mbox{ (universal concept)}\\
\bot^{\I}          & = & \emptyset & \mbox{ (bottom concept)}\\
(\neg C)^{\I} & = & \Delta^{\I} \setminus C^{\I} & \mbox{  (concept negation)}\\
(C \sqcup D)^{\I} & = & C^{\I} \cup D^{\I} & \mbox{ (concept union)}\\
(C \sqcap D)^{\I} & = & C^{\I} \cap D^{\I} & \mbox{ (concept intersection)}\\
\{a\}^{\I} & = & \{a^{\I}\} & \mbox{ (nominal)}\\
(\selfrestriction{R})^{\I} & = & \{x \in \Delta^{\I}: [x,x] \in R^{\I}\} & \mbox{ (self restriction)}\\

(\forall R.C)^{\I} & = &
\{x \in \Delta^{\I} : (\forall [x,y] \in R^{\I})(y \in C^{\I}) \}
& \mbox{ (value restriction)}\\

(\exists R.C)^{\I}& = & \{x \in \Delta^{\I} : (\exists y \in C^{\I})([x,y] \in R^{\I}) \}
%& \mbox{ (existential quantification)}\\
&\mbox{ (existential quantifier)}\\
(\leq n R.C)^{\I} & = & \{x \in \Delta^{\I} : |\{y \in C^{\I} : [x,y] \in R^{\I} \}| \leq n\} & \mbox{ (number restrictions)}\\
(\geq n R.C)^{\I} & = & \{x \in \Delta^{\I} : |\{y \in C^{\I} : [x,y] \in R^{\I} \}| \geq n\} &\\
(R \subseteq S)^{\I} & = & \{x \in \Delta^{\I} : (\forall y \in \Delta^{\I})([x,y] \in R^{\I} \rightarrow &\\
  &&\;\;\;\;\;[x,y] \in S^{\I})\}& \mbox{ (role-value-map)}\\
&&&\\
\hline
&&&\\
\universalrole^{\I} & = & \Delta^{\I} \times \Delta^{\I} & \mbox{ (universal role)}\\
(\neg R)^{\I} & = & (\Delta \times \Delta) \setminus R^{\I} & \mbox{  (role negation)}\\
(R \sqcup S)^{\I} & = & R^{\I} \cup S^{\I} & \mbox{ (role union)}\\
(R \sqcap S)^{\I} & = & R^{\I} \cap S^{\I} & \mbox{ (role intersection)}\\
(R^{-})^{\I} & = & \{[x,y] \in \Delta^{\I}\times\Delta^{\I} : [y,x] \in  R^{\I}\} & \mbox{ (role inverse)}\\
(R_{C|})^{\I} & = & \{ [x,y] \in R^{\I} : x \in C^{\I} \} & \mbox{ (role restrictions)}\\
(R_{|D})^{\I} & = & \{ [x,y] \in R^{\I} : y \in D^{\I} \} & \\
(R_{C|D})^{\I} & = & (R_{C|})^{\I} \cap (R_{|D})^{\I} & \\

id(C)^{\I} & = & \{ [x,x] : x \in C^{\I} \} & \mbox{ (role identity)}\\
(R \circ S)^{\I} & = & R^{\I} \circ S^{\I} & \mbox{ (role composition)}\\
(R^{*})^{\I} & = & (R^{\I})^{*} & \mbox{ (transitive closure)}\\
(\symmetricclosure{R})^{\I} & = & R^{\I} \cup (R^{-})^{\I} & \mbox{ (symmetric closure)}\\
&&&\\
\hline
&&&\\
\I \models C \Issub D & \iff & C^{\I} \subseteq D^{\I} & \mbox{ (inclusion axioms)}\\
\I \models R \Issub S & \iff & R^{\I} \subseteq S^{\I}&\\
\I \models C \equiv D & \iff & C^{\I} = D^{\I}& \mbox{ (equivalence axioms)}\\
\I \models R \equiv S & \iff & R^{\I} = S^{\I}&\\
&&&\\
\hline
&&&\\
\I \models \istransitive{R} & \iff & R^{\I} \circ R^{\I} \subseteq R^{\I}&\mbox{ (role transitivity)}\\
\I \models \isreflexive{R} & \iff & (\roleidentity{\exists R.\top})^{\I} \subseteq R^{\I} & \mbox{ (role reflexivity)}\\
\I \models \isantisymmetric{R} & \iff & R^{\I} \cap (R^{-})^{\I} = \emptyset & \mbox{ (role asymmetry)}\\
&&&\\
\hline
&&&\\
\I \models C(a) & \iff & a^{\I} \in C^{\I}& \mbox{ (concept assertion)}\\
\I \models R(a,b) & \iff & [a^{\I}, b^{\I}] \in R^{\I}& \mbox{ (role assertion)}\\
&&&\\
\hline
\end{array}
\end{math}
\caption{Description logic constructs}\label{DLCONS}
\end{table}


It turns out that the semantical definitions
of several description logic statements $\Sigma$ may be expressed as formulae of
the form
\[
 \I \models \Sigma \mbox{ iff } \left(\forall x_1 \in \Delta^{\I}\right)
  \ldots \left(\forall x_n \in \Delta^{\I}\right)\Gamma_\Sigma,
\]
where $\Gamma_\Sigma$ is a Boolean combination of expressions of the types
\[
u \in C^{\I},~ [u,u'] \in R^{\I},~ u=a^{\I},~ u=u',
\]
with $C,R,a$ respectively a concept term, a role term, and an
individual name, and with $u,u'$ ranging over the variables $x_1,
\ldots, x_n$ (see Table \ref{DLCONS}).

This holds in particular for all the knowledge base statements
allowed in the novel description logic \dlLang defined next.

\begin{definition}\label{DLLANG}
Let $\mathcal{N}^c, \mathcal{N}^r, \mathcal{N}^i$ be the three
denumerable, infinite and mutually disjoint collections of, respectively,
concept, role, and individual names.
\dlLang-concept terms and \dlLang-role terms are formed according to the
following syntax rules:
\[
 \begin{array}{rcl}
  C,D & \longrightarrow & 
  A \,|\, \top \,|\, \bot \,|\, \neg C \,|\, C \sqcup D \,|\, C \sqcap D \,|\, \{ a \} \,|\, \selfrestriction{R} \,|\, \exists R.\{a\}\\
%
  R,S & \longrightarrow & P \,|\, \universalrole \,|\, R^{-} \,|\, \neg R \,|\, R \sqcup S \,|\, R \sqcap S \,|\, R_{C|} \,|\,R_{|D} \,|\, R_{C|D} \,|\, \roleidentity{C} \,|\, \symmetricclosure{R}
 \end{array}
\]
where $C,D$ denote \dlLang-concept terms, $R,S$ denote \dlLang-role
terms, $A, P$ denote a concept and a role name, respectively, and $a$
denotes an individual name.  A \dlLang-knowledge base is then a finite
collection of statements of the following types:
\[
 \begin{array}{lllll}
 C \equiv D\;, & C \Issub D\;, & R \equiv S\;, &R \Issub S\;, & C \Issub \forall R.D\;, \\
 \exists R.C \Issub D\;, & R \circ R' \Issub S\;, & \istransitive{R}\;, & \isreflexive{R} \;, &\isantisymmetric{R}
\end{array}
\]
where $C,D$ are \dlLang-concept terms and $R,S,R'$ are \dlLang-role terms.
\end{definition}

Notice that the above definition of \dlLang is not minimal,
as we intended to give a clear and immediate overview of its
expressive power.

The major limitation of \dlLang (with respect to other description
logics) is that value restriction and existential quantification are
restricted to the left-hand side and right-hand side of inclusions,
respectively.  Moreover, number restrictions are not allowed.  On
the other hand, the set of allowed constructs is extremely large.  In
particular, complex role constructors can be used freely, in contrast
with most expressive description logics.  Additionally, reasoning in
\dlLang is \textsc{NP}-complete, as will be proved in the following
theorem.

\begin{theorem}\label{DLANGT}
The consistency problem for \dlLang-knowledge bases is \textsc{NP}-complete.
\end{theorem}
\begin{proof}
We will show that the consistency problem for \dlLang-knowledge bases
reduces to the satisfiability problem for \LangBounded{2}-formulae.

We begin with observing that we can restrict our attention to
\dlLang-knowledge bases containing only statements of the following
types:
\[
\begin{array}{lllllll}
  A \equiv \top \;,& A \equiv \neg B \;,& A \equiv B \sqcup B' \;,& A \equiv \{a\} \;,& A \Issub \forall P.B \;,& \exists P.A \Issub B \;,& A \equiv \exists P.\{a\},\\
  P \equiv \universalrole \;,& P \equiv \neg Q \;,& P \equiv Q \sqcup Q' \;,& P \equiv Q^{-} \;,& P \equiv \roleidentity{A} \;,& P \equiv Q_{A|} \;,& P \circ P' \Issub Q,\\
  \isreflexive{P}
\end{array}
\]
where $A,B,B'$ are concept names, $P,P',Q,Q'$ are role names, and $a$ is
an individual name, since any \dlLang-knowledge base $\mathcal{K}$ can
be easily transformed into a knowledge base $\mathcal{K'}$ which
contains only statements of these types, and such that $\mathcal{K}$
is consistent if and only if $\mathcal{K'}$ is.

Next, we define a mapping $\tau$ from \dlLang-statements to simple
prenex \Lang-formulae as follows:
\[
 \begin{array}{rcl}
  \tau(A \equiv \top) & \defAs & \left(\forall x \in \nonpairs{\Delta}\right)\left(x \in \nonpairs{A}\right)\\
  \tau(A \equiv \neg B) & \defAs & \left(\forall x \in \nonpairs{\Delta}\right)\left(x \in \nonpairs{A} \leftrightarrow x \notin \nonpairs{B}\right)\\
  \tau(A \equiv B \sqcup B') & \defAs & \left(\forall x \in \nonpairs{\Delta}\right)\left(x \in \nonpairs{A} \leftrightarrow x \in \nonpairs{B} \vee x \in \nonpairs{B'}\right)\\
  \tau(A \equiv \{a\}) & \defAs & \left(\forall x \in \nonpairs{\Delta}\right)\left(x \in \nonpairs{A} \leftrightarrow x=a\right) \wedge a \in \nonpairs{A}\\
  \tau(A \Issub \forall P.B) & \defAs & \left(\forall [x,y] \in P\right)\left(x \in \nonpairs{A} \rightarrow y \in \nonpairs{B}\right)\\
  \tau(\exists P.A \Issub B) & \defAs & \left(\forall [x,y] \in P\right)\left(y \in \nonpairs{A} \rightarrow x \in \nonpairs{B}\right)\\
  \tau(A \equiv \exists P.\{a\}) & \defAs & \left(\forall x \in \nonpairs{\Delta}\right)\left(x \in \nonpairs{A} \leftrightarrow [x,a] \in P\right)\\
  \tau(P \equiv \universalrole) & \defAs & \left(\forall [x,y] \in \Delta\right)\left([x,y] \in P\right)\\
  \tau(P \equiv \neg Q) & \defAs & \left(\forall x,y \in \nonpairs{\Delta}\right)\left([x,y] \in P \leftrightarrow [x,y] \notin Q\right)\\
  \tau(P \equiv Q \sqcup Q') & \defAs & \left(\forall x,y \in \nonpairs{\Delta}\right)\left([x,y] \in P \leftrightarrow [x,y] \in Q \vee [x,y] \in Q'\right)\\
  \tau(P \equiv Q^{-}) & \defAs & \left(\forall x,y \in \nonpairs{\Delta}\right)\left([x,y] \in P \leftrightarrow [y,x] \in Q\right)\\
  \tau(P \equiv Q_{A|}) & \defAs & \left(\forall x,y \in \nonpairs{\Delta}\right)\left([x,y] \in P \leftrightarrow [x,y] \in Q \wedge x \in \nonpairs{A}\right)\\
  \tau(P \equiv \roleidentity{A}) & \defAs & \left(\forall x,y \in \nonpairs{\Delta}\right)\left([x,y] \in P \leftrightarrow x=y \wedge x \in \nonpairs{A}\right)\\
  \tau(P \circ P' \Issub Q) & \defAs & \left(\forall [x,y]\in P\right)\left(\forall [y',z] \in P'\right)\left(y=y' \rightarrow [x,z] \in Q\right)\\
  \tau(\isreflexive{P}) & \defAs & \left(\forall [x,y] \in P\right)\left( [x,x] \in P\right)\\
 \end{array}
\]
We remark that in the above definition of the mapping $\tau$ we are
assuming that the collection $\Vars$ of the variables of the language
\Lang contains all the concept, role, and individual names.  Moreover,
we used the same symbol $\Delta$ which is also used to denote the
domain of a description logic interpretation, under the assumption
that $\Delta \notin \mathcal{N}^c \cup \mathcal{N}^r \cup
\mathcal{N}^i$.  These are just technical assumptions (not strictly
necessary for the proof) which have been just introduced to enhance
readability of the formulae $\tau(\cdot)$ and to emphasize the strong
correlation between the semantical definitions of \dlLang-statements
and their corresponding \Lang-formulae.

Now let $\mathcal{K}$ be a \dlLang-knowledge base.  We define the
\Lang-formula $\varphi$, expressing the consistency of $\mathcal{K}$,
as follows
\[
\begin{array}{lcl}
 \varphi & \defAs &  \varphi_{\Delta} \wedge \varphi_C \wedge \varphi_R \wedge \varphi_I \wedge \varphi_{\mathcal{K}}\\
 \varphi_{\Delta} & \defAs & \left(\forall [x,y] \in \Delta\right)\left([x,y] \notin \Delta\right)\\
 \varphi_C & \defAs & \bigwedge\limits_{A \in \mathsf{Cpts}} \left( \left(\forall x \in \nonpairs{A}\right)\left(x \in \nonpairs{\Delta}\right) \wedge
\left(\forall [x,y] \in A\right)\left([x,y] \notin A\right)\right)\\
 \varphi_R & \defAs & \bigwedge\limits_{P \in \mathsf{Rls\phantom{s}}} \left( \left(\forall x \in \nonpairs{P}\right)\left(x \notin \nonpairs{P}\right) \wedge
\left(\forall [x,y] \in P\right)\left(x \in \nonpairs{\Delta} \wedge y \in \nonpairs{\Delta}\right)\right)\\
 \varphi_I & \defAs & \bigwedge\limits_{a \in \mathsf{Inds}} a \in \nonpairs{\Delta}\\
  \varphi_{\mathcal{K}} & \defAs & \bigwedge\limits_{\Sigma \in \mathcal{K}} \tau(\Sigma)
\end{array}
\]
where $\mathsf{Cpts}, \mathsf{Rls}$, and $\mathsf{Inds}$ are
respectively the sets of concept, role and individual names occurring
in $\mathcal{K}$.

The consistency problem for $\mathcal{K}$ is equivalent to the
satisfiability of $\varphi$, as we prove next.
% We prove that $\varphi$ is satisfiable if and only if $\mathcal{K}$ is
% consistent.

Plainly, $\varphi_{\Delta}, \varphi_{C}, \varphi_{R}$, and $\varphi_{I}$
guarantee that each model of $\varphi$ can be easily turned into a
\dlLang-interpretation. Additionally, $\varphi_{\mathcal{K}}$ ensures
that the  \dlLang-interpretation obtained in this way satisfies all
the statements in $\mathcal{K}$.

Conversely, let \I be a model for $\mathcal{K}$. Without loss of
generality, we may assume that $\Delta^{\I}$ is a set belonging to
the von Neumann hierarchy $\VNU$ (otherwise, we embed $\Delta^{\I}$ in
 $\VNU$). Let $\inter=(\assignment{\inter}, \pairf{\inter})$ be the \Lang-interpretation,
induced by \I, defined by
\[
\begin{array}{rcll}
 \pairf{\inter}(u,v) & \defAs & \{u, \{u,v\}, \Delta^{\I}\} & \mbox{ for all }u,v \in \VNU\\
 \assignment{\inter} \Delta & \defAs & \Delta^{\I}&\\
 \assignment{\inter} A & \defAs & A^{\I} & \mbox{ for all }A \in \mathcal{N}^c\\
 \assignment{\inter} P & \defAs &\{ \pairf{\inter}(u,v) : [u,v] \in P^{\I}\} & \mbox{ for all }P \in \mathcal{N}^r\\
 \assignment{\inter} a & \defAs & a^{\I} & \mbox{ for all }a \in \mathcal{N}^i.
\end{array}
\]
Since $\inter \Delta \in \pairf{\inter}(u,v)$ for all $u,v \in \VNU$,
from the well-foundedness of the membership relation it follows that
$\inter \Delta$ does not contain any pair (with respect to
$\pairf{\inter}$).  Thus $x \in \inter \nonpairs{A} \iff x \in A^{\I}$
and $\pairf{\inter}(x,y) \in \inter P \iff [x,y] \in P^{\I}$ follow
from the definition of $\inter$, and then $\inter \tau(\Sigma) =
\true$ if and only if $\I$ satisfies $\Sigma$, for all the statements
$\Sigma \in \mathcal{K}$.

We conclude the proof by observing that each conjunct in $\varphi$
contains at most two quantifiers (i.e., $\varphi$ is a formula of
\LangBounded{2}), thus in view of Corollary~\ref{UDECSIZEBOUNDED} the
satisfiability of $\varphi$ can be checked in nondeterministic
polynomial time, while the \textsc{NP}-hardness of this problem
follows directly from the \textsc{NP}-completeness of the
satisfiability problem for propositional formulae.
\end{proof}

\section{Extending \dlLang with \textsf{SWRL} rules}\label{SWRL}

In order to increase the expressive power of description logics, in
\cite{HorPat2004} it was proposed to extend this framework with a
simple form of Horn-style rules called \textsf{SWRL} rules.  \textsf{SWRL}
rules have the form
\[
 H \rightarrow B_1 \wedge \ldots \wedge B_n
\]
where $H, B_1, \ldots, B_n$ are \emph{atoms} of the forms $A(x),
P(x,y), x=y, x\neq y$, with $A$ a concept name, $P$ a role
name, and $x,y$ either \textsf{SWRL}-variables or individual names.

A \emph{binding} $\mathcal{B}(\I)$ is any extension of
the interpretation $\I$ which assigns a domain item to each
\textsf{SWRL}-variable. An interpretation $\I$ satisfies a rule
$H \rightarrow B_1 \wedge \ldots \wedge B_n$ if each
binding $\mathcal{B}(\I)$ which satisfies all the atoms
$B_1, \ldots, B_n$ satisfies $H$ also.

A \dlLang-knowledge base $\mathcal{K}$ extended with
a finite set of \textsf{SWRL} rules $\mathcal{R}$ is said to be
\emph{satisfiable} if and only if it has a model
which satisfies all the rules in $\mathcal{R}$.

The reduction provided in Section \ref{DL} can be
easily extended to cope with \dlLang-knowledge bases extended
with finite sets of \textsf{SWRL} rules, as shown in the following theorem.

\begin{theorem}
The consistency problem for \dlLang-knowledge bases extended with
finite sets of \textsf{SWRL} rules is decidable.
\end{theorem}
\begin{proof}
Let $\mathcal{K}$ be a \dlLang-knowledge base, and let
$\mathcal{R}$ be a finite set of \textsf{SWRL} rules. Let us extend
the mapping $\tau$, defined in Theorem \ref{DLANGT},
to \textsf{SWRL} rules and atoms as follows:
\[
 \begin{array}{rcl}
  \tau(H \rightarrow B_1 \wedge \ldots \wedge B_n) & \defAs&
  \left(\forall x_1, \ldots, x_m \in \nonpairs{\Delta}\right)\left(\tau(H) \rightarrow \tau(B_1) \wedge \ldots
  \wedge \tau(B_n)\right)\\
%
  \tau(A(x)) & \defAs & x \in \nonpairs{A}\\
%
  \tau(P(x,y)) & \defAs & [x,y] \in P \\
%
  \tau(x=y) & \defAs & x=y\\
%
  \tau(x\neq y) & \defAs & x\neq y\\
 \end{array}
\]
where $H, B_1, \ldots B_n$ are \textsf{SWRL} atoms, $x_1, \ldots, x_m$ are the \textsf{SWRL} variables occurring
in $H \rightarrow B_1 \wedge \ldots \wedge B_n$, $x,y$ can be either \textsf{SWRL} variables or
individual names, and $A,P$ are respectively
a concept and a role name.

We conclude the proof by observing that the following \Lang-formula
$\varphi'$ is satisfiable if and only if the knowledge base
$\mathcal{K}$ extended with $\mathcal{R}$ is consistent:
\[
 \varphi' \defAs \bigwedge\limits_{\rho \in \mathcal{R}}\tau(\rho)
 \wedge \varphi,
\]
where $\varphi$ is built from $\mathcal{K}$ as described in
Theorem~\ref{DLANGT}, extending $\mathsf{Cpts}$, $\mathsf{Rls}$ and
$\mathsf{Inds}$ with the concept, role and individual names occurring
in $\mathcal{R}$, respectively.
\end{proof}

\section{Conclusions and future works}\label{CONCLUSIONS}

We have introduced the collection of quantified \Lang-formulae of set
theory, which allow the explicit manipulation of ordered pairs, and
proved that they have a decidable satisfiability problem.  In fact,
when restricted to \Lang-formulae whose conjuncts have quantifier
prefixes of length bounded by a constant, the satisfiability problem
is \textsc{NP}-complete.

In addition, we have introduced the novel description logic \dlLang
and shown that its consistency check is \textsc{NP}-complete, since it
can be reduced to the satisfiability test for a \Lang-formula whose
conjuncts involve at most two quantifiers.  Finally we have extended
the description logic \dlLang with \textsf{SWRL} rules without
disrupting the decidability of the knowledge base consistency problem.

In contrast with description logics, 
the semantics of set theory is \emph{multi-level}, so that
sets (and consequently relations) can be nested
arbitrarily.
In the light of this observation, we intend to
investigate whether the description logic \dlLang can be extended with
\emph{meta-modeling}
features (cf.
\cite{Mot2007}), which would allow to state relationships
among elements of the conceptual model.

Finally, we intend to investigate if \Lang (and consequently \dlLang) can be
extended with concrete domains, in order to promote definitively \Lang
as a language for knowledge representation, and, consequently,
for the semantic web.

\begin{small}
\begin{thebibliography}{10}

\bibitem{DLHANDBOOK}
Franz Baader, Diego Calvanese, Deborah~L. McGuinness, Daniele Nardi, and
  Peter~F. Patel-Schneider, editors.
\newblock {\em The Description Logic Handbook: Theory, Implementation, and
  Applications}.
\newblock Cambridge University Press, 2003.

\bibitem{BreFerOmoSch1981}
Michael Breban, Alfredo Ferro, Eugenio~G. Omodeo, and Jacob~T. Schwartz.
\newblock Decision procedures for elementary sublanguages of set theory. {II}.
  {F}ormulas involving restricted quantifiers, together with ordinal, integer,
  map, and domain notions.
\newblock {\em Communications on Pure and Applied Mathematics}, 34:177--195,
  1981.

\bibitem{CanFer1995}
Domenico Cantone and Alfredo Ferro.
\newblock {\em Techniques of computable set theory with applications to proof
  verification}, volume XLVIII of {\em Comm. Pure Appl. Math.}, pages 901--945.
\newblock Wiley, 1995.

\bibitem{CanFerOmo89a}
Domenico Cantone, Alfredo Ferro, and Eugenio Omodeo.
\newblock {\em Computable set theory}, volume~6 of {\em International Series of
  Monographs on Computer Science}.
\newblock Oxford Science Publications. Clarendon Press, Oxford, {UK}, 1989.

\bibitem{CanLonNic2010}
Domenico Cantone, Cristiano Longo, and Marianna {Nicolosi Asmundo}.
\newblock A {D}ecision {P}rocedure for a {T}wo-sorted {E}xtension of
  {M}ulti-{L}evel {S}yllogistic with the {C}artesian {P}roduct and {S}ome {M}ap
  {C}onstructs.
\newblock In Wolfgang Faber and Nicola Leone, editors, {\em CILC2010 : 25th
  Italian Conference on Computational Logic}, 2010.

\bibitem{CanLonPis2010}
Domenico Cantone, Cristiano Longo, and Antonio Pisasale.
\newblock {C}omparing {D}escription {L}ogics with {M}ulti-level {S}yllogistics:
  the {D}escription {L}ogic \dlmlsscart.
\newblock In {\em 6th Workshop on Semantic Web Applications and Perspectives
  (SWAP)}, 2010.

\bibitem{CanOmoPol01}
Domenico Cantone, Eugenio Omodeo, and Alberto Policriti.
\newblock {\em Set theory for computing: from decision procedures to
  declarative programming with sets}.
\newblock Monographs in Computer Science. Springer-Verlag, New York, NY, USA,
  2001.

\bibitem{CanOmoSchUrs03}
Domenico Cantone, Eugenio~G. Omodeo, Jacob~T. Schwartz, and Pietro Ursino.
\newblock Notes from the logbook of a proof-checker's project.
\newblock In Nachum Dershowitz, editor, {\em Verification: Theory and
  Practice}, volume 2772 of {\em Lecture Notes in Computer Science}, pages
  182--207. Springer, 2003.

\bibitem{CanSch91}
Domenico Cantone and Jacob~T. Schwartz.
\newblock {D}ecision {P}rocedures for {E}lementary {S}ublanguages of {S}et
  {T}heory: {XI}. {M}ultilevel {S}yllogistic {E}xtended by {S}ome {E}lementary
  {M}ap {C}onstructs.
\newblock {\em J. Autom. Reasoning}, 7(2):231--256, 1991.

\bibitem{FerOmoSch1980}
Alfredo Ferro, Eugenio~G. Omodeo, and Jacob~T. Schwartz.
\newblock {\em Decision Procedures for Elementary Sublanguages of Set Theory.
  I. Multi-level syllogistic and some extensions.}, volume XXXIII of {\em Comm.
  Pure Appl. Math.}, pages 599--608.
\newblock Wiley, 1980.

\bibitem{HorKutSat2006}
Ian Horrocks, Oliver Kutz, and Ulrike Sattler.
\newblock {The Even More Irresistible SROIQ}.
\newblock In {\em Proc. of the 10th Int. Conf. on Principles of Knowledge
  Representation and Reasoning (KR2006)}, pages 57--67. 10th International
  Conference on Principles of Knowledge Representation and Reasoning, AAAI
  Press, June 2006.

\bibitem{HorPat2004}
Ian Horrocks and Peter~F. Patel-Schneider.
\newblock A proposal for an {OWL} rules language.
\newblock In Stuart~I. Feldman, Mike Uretsky, Marc Najork, and Craig~E. Wills,
  editors, {\em WWW}, pages 723--731. ACM, 2004.

\bibitem{KroRudHit2008}
Markus Kr{\"o}tzsch, Sebastian Rudolph, and Pascal Hitzler.
\newblock Description {L}ogic {R}ules.
\newblock In Malik Ghallab, Constantine~D. Spyropoulos, Nikos Fakotakis, and
  Nikolaos~M. Avouris, editors, {\em ECAI}, volume 178 of {\em Frontiers in
  Artificial Intelligence and Applications}, pages 80--84. IOS Press, 2008.

\bibitem{Lutz2001}
Carsten Lutz and Ulrike Sattler.
\newblock The {C}omplexity of {R}easoning with {B}oolean {M}odal {L}ogics.
\newblock In Frank Wolter, Heinrich Wansing, Maarten de~Rijke, and Michael
  Zakharyaschev, editors, {\em Advances in Modal Logic}, pages 329--348. World
  Scientific, 2000.

\bibitem{Mot2007}
Boris Motik.
\newblock On the {P}roperties of {M}etamodeling in {OWL}.
\newblock {\em J. Log. Comput.}, 17(4):617--637, 2007.

\bibitem{MotSatStu2005}
Boris Motik, Ulrike Sattler, and Rudi Studer.
\newblock Query {A}nswering for {OWL-DL} with rules.
\newblock {\em J. Web Sem.}, 3(1):41--60, 2005.

\bibitem{OmoCanPolSch06}
Eugenio Omodeo, Domenico Cantone, and Alberto Policriti.
\newblock Reasoning, {A}ction and {I}nteraction in {AI} {T}heories and
  {S}ystems, {E}ssays {D}edicated to {L}uigia {C}arlucci {A}iello.
\newblock In Oliviero Stock and Marco Schaerf, editors, {\em Reasoning, Action
  and Interaction in AI Theories and Systems}, volume 4155 of {\em Lecture
  Notes in Computer Science}. Springer, 2006.

\bibitem{ParPol93}
Franco Parlamento and Alberto Policriti.
\newblock Undecidability results for restricted universally quantified formulae
  of set theory.
\newblock {\em Comm. Pure Appl. Math.}, {ILVI}:57--73, 1993.

\bibitem{SchwCanOmoPol11}
Jacob~T. Schwartz, Domenico Cantone, and Eugenio~G. Omodeo.
\newblock {\em Computational {L}ogic and {S}et {T}heory: {A}pplying
  {F}ormalized {L}ogic to {A}nalysis}.
\newblock Texts in Computer Science. Springer-Verlag New York, Inc., 2011.

\bibitem{SchDewSchDub1986}
Jacob~T. Schwartz, Robert B.~K. Dewar, Edmond Schonberg, and E~Dubinsky.
\newblock {\em Programming with sets; an introduction to SETL}.
\newblock Springer-Verlag New York, Inc., New York, NY, USA, 1986.

\end{thebibliography}
\end{small}

\end{document}
